\section{Introduction}
Nowadays what makes a software product stand out is less technological and more related to how it handles the human computer interaction. This topic has been highly discussed in the software engineering community in the past few years.

Even with all the available tools and comprehensive bibliography it's still hard to build good user interfaces. In other fields of engineering there's a set of tested and robust patterns that can be used to build good products. In software engineering there's already a set of patterns for software architects to design their applications, but there's still little use of patterns in user interfaces.

This work needs a way to describe and store a set of patterns. UsiXML and UsiPXML are two languages that fit the requirements and were studied for this paper.
 
UsiXML is a user interface description language aimed at expressing user interfaces built with various modalities of interaction and independently of them. UsiXML is XML compliant to enable flexible exchange of information and powerful communication between models and tools used in user interface engineering \cite{UsiXML_USer_Interface_eXtensible_Markup_Language}. UsiXML has a specification in UML that we'll see in detail on section \ref{subsection:UsiXML}. One of the great advantages of UsiXML is platform independence providing a multi-path development of user interfaces \cite{UsiXML_a_Language_Supporting_Multi-Path_Development_of_User_Interfaces}.


UsiPXML resulted from merging two languages, PLML \cite{Pattern_Language_Markup_Language} and UsiXML \cite{Different_kinds_of_pattern_support_for_interactive_systems}. Pattern Language Markup Language (PLML) was introduced as an attempt to uniformly represent user interface patterns. It's natural language based so it suffers from intrinsic problems like ambiguity and inconsistency.

The main goal of this work is to study and develop a tool that can interpret a set of patterns specified in either UsiXML or UsiPXML, link them to the source code of the business layer through code annotations and generate a user interface based on the implemented functionality and the base pattern. This can help developers to build user interfaces with little effort and little knowledge on human computer interaction only by using good patterns that have been tested and are known to be robust and compliant with HCI rules.

This paper serves as an initial study for the development of a tool that generates user interfaces based on patterns. Section \ref{section:How_user_interfaces_are_built} explains how developers currently build their user interfaces and identifies the need of better tool support on this area. The third section gives some theory in the field of patterns engineering. On section \ref{section:How_to_specify_user_interface_patterns} we'll see some languages capable of successfully describe user interface patterns, namely UsiXML and UsiPXML and a light initial specification of the proposed tool. The last section of this paper handles the conclusions of this study.