\section{Introduction}
Nowadays what makes a software product stand out is less technological and more related to how it handles the human computer interaction. This topic has been highly discussed in the software engineering community in the past few years.

Even with all the available tools and comprehensive bibliography it's still hard to build good user interfaces. In other fields of engineering there's a set of tested and robust patterns that can be used to build good products. In software engineering there's already a set of patterns for software architects to design their applications. But there's still little use of patterns in user interfaces.

This work needs a way to describe and store a set of patterns. UsiXML is an XML based language that was created with the purpose of specifying user interfaces for multiple contexts. UsiPXML is a language that combines UsiXML with PLML whish is a language for describing user interface patterns in a more descriptive way. Both these languages seem to be adequate to describe and store a collection of patterns as intended. The former one is more concise while the latter one can store information in a more structured way.

The main goal of this work is to study and develop a tool that can interpret a set of patterns specified in either UsiXML or UsiPXML, link them to the source code of the business layer through code annotations and generate a user interface based on the implemented functionality and the base pattern. This can help developers to build user interfaces with little effort and little knowledge on human computer interaction only by using good patterns that have been tested and are known to be robust and compliant with HCI rules.