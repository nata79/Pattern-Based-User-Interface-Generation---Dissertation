\subsection{Manually coding user interfaces}
Before more advanced tools were created, user interfaces had to be manually coded. Although the evolution that the industry has suffered since the beginnings, most developers still prefer to code manually user interfaces because it gives them more control over their work.

This is a very time consuming technique because humans have to do most of the work but in the end it really depends on what language or framework you're working on. Most popular and modern programming languages give developers access to frameworks for building graphical user interfaces (GUI) like GTK+\footnote{http://www.gtk.org/}, Swing\footnote{http://docs.oracle.com/javase/tutorial/uiswing/} or Windows Forms\footnote{http://msdn.microsoft.com/en-us/library/dd30h2yb.aspx}. These examples are for the desktop side. On the web side everything is (X)HTML, CSS and JavaScript but there are a lot of frameworks to abstract from these languages like JSF\footnote{http://www.oracle.com/technetwork/java/javaee/javaserverfaces-139869.html}, Struts\footnote{http://struts.apache.org/} or ASP.net\footnote{http://www.asp.net/}.

Most desktop GUI frameworks use the same language for views and the other software layers. This means that a lot of code has to be written in order to get things done. Frameworks like GTK+, Swing and Windows forms are very hard to use without help from more advanced tools.

Figure \ref{content/how_user_interfaces_are_built/login_window.png} shows a simple login window with two textboxes for username and password and a couple of buttons to login or cancel the operation.
\image{6cm}{content/how_user_interfaces_are_built/login_window.png}{Login window}

On listing \ref{Login_window_using_Swing_coded_manually} we can see the code written to build this window using Swing. Additional information about swing can be found in \cite{Java_Swing}

\lstset{language=Java}
\begin{lstlisting}[caption={Login window using Swing, coded manually},label=Login_window_using_Swing_coded_manually]
public static void main(String[] args) {
        SwingManualTest sm = new SwingManualTest();
        sm.showLoginWindow();
    }

    private void showLoginWindow(){
        Container c = getContentPane();
        c.setLayout(new GridLayout(3, 2));
        c.add(new JLabel("Username:"));
        c.add(new JTextField());
        c.add(new JLabel("Password:"));
        c.add(new JTextField());
        c.add(new JButton("Login"));
        c.add(new JButton("Cancel"));
        setDefaultCloseOperation(JFrame.EXIT_ON_CLOSE);
        pack();
        setVisible(true);
    }
\end{lstlisting}
It's plain Java so every control is an object. For some object oriented programming (OOP) enthusiasts this is a good thing but it's incomprehensible for designers and even for most developers this is very hard and thus very time consuming.

Fortunately, on the web side things are simpler. Most frameworks use HTML with some specific extensions to specify the views. This offer developers a more declarative paradigm which makes a lot more sense when building interfaces. This approach also produces a lot less code which makes maintenance a lot easier. Listing \ref{Login_window_using_JSF} shows the code to build the login window using JSF. Additional information about JSF can be found in \cite{Mastering_JavaServer_Faces}.
\lstset{language=HTML}
\begin{lstlisting}[caption={Login window using JSF}, label=Login_window_using_JSF]
<html xmlns="http://www.w3.org/1999/xhtml"
      xmlns:h="http://java.sun.com/jsf/html">
    <h:head>
        <title>Login</title>
    </h:head>
    <h:body>
        <h:form>
            <h:outputLabel value="Username:" />
            <h:inputText />
            <h:outputLabel value="Password:" />
            <h:inputSecret />
            <h:button value="Login" />
            <h:button value="Cancel" />
        </h:form>
    </h:body>
</html>
\end{lstlisting}
This is very different from the first example. With JSF, views are specified in a specific language. This makes the code more readable and easier to maintain.

Recently have been developed new frameworks for desktop GUI's that resemble the web ones referenced earlier. One good example is WPF\footnote{http://msdn.microsoft.com/en-us/library/aa970268.aspx}. It uses XAML\footnote{http://msdn.microsoft.com/en-us/library/ms752059.aspx} to specify views. It's a markup language based on XML and, thus, more similar to HTML than a programming language like Java or C\#. Let's take a look at the login window coded for WPF. Additional information about WPF and XAML can be found in \cite{WPF_in_Action_with_Visual_Studio_2008}.
\lstset{language=XML}
\begin{lstlisting}[caption={Login window using WPF}]
<Window x:Class="WpfApplication1.MainWindow"
        xmlns="http://schemas.microsoft.com/winfx/2006/xaml/presentation"
        xmlns:x="http://schemas.microsoft.com/winfx/2006/xaml"
        Title="Login Window" >
    <Grid>
        <Grid.ColumnDefinitions>
            <ColumnDefinition />
            <ColumnDefinition />
        </Grid.ColumnDefinitions>
        <Grid.RowDefinitions>
            <RowDefinition />
            <RowDefinition />
            <RowDefinition />
        </Grid.RowDefinitions>
        <TextBlock Text="Username:" Grid.Column="0" Grid.Row="0" />
        <TextBox Grid.Column="1" Grid.Row="0" Width="150" />
        <TextBlock Text="Password:" Grid.Column="0" Grid.Row="1" />
        <TextBox Grid.Column="1" Grid.Row="1" Width="150" />
        <Button Grid.Column="0" Grid.Row="2" Width="150">Login</Button>
        <Button Grid.Column="1" Grid.Row="2" Width="150">Cancel</Button>
    </Grid>
</Window>
\end{lstlisting}
Even though this language is a lot more verbose than HTML and other markup languages it's a very good alternative for building desktop GUI's, especially if you're going to write all the code manually.

The conclusion of this section is that manually coding user interfaces isn't always a good idea depending on the technology you're using. The first frameworks that were presented use programming languages to specify the views. That doesn't look like a very good approach because it's not intuitive for the developer and incomprehensible for designers. On the other hand the later solutions use specific languages for specifying views which are more intuitive and easy to write but they force developers to learn these new languages. The other problem is that all the code produced is platform specific. If you're planning on porting your application to other devices, all the code has to be written all over again.
