\subsection{Model driven development of user interfaces}
Model driven development defining characteristic is that software development's primary focus and products are models rather than computer programs. The major advantage of this is that we express models using concepts that are much less bound to the underlying implementation technology and are much closer to the problem domain relative to most popular programming languages \cite{The_Pragmatics_of_Model-Driven_Development}.

Models are easier to maintain than the code itself and, most important, they're platform independent. This means that the same model can be used to generate code that runs on a desktop environment, a web environment or even a mobile environment. This makes a lot of sense for user interfaces because modern applications are becoming more and more ubiquitous and it's highly complex and time consuming to build a GUI for every supported platform.

UML\footnote{Unified Modelling Language} is the industry standard for software modelling but, unfortunately, is not fit to model user interfaces. With this in mind, the software engineering community has developed some new modelling languages in the past few years to overcome this problem. The most relevant are probably UMLi\footnote{Unified Modelling Language for Interactive Applications}, an extension to UML and CTT\footnote{Concur Task Trees} which aims task modelling. UMLi provides an alternative diagram notation for describing abstract interaction objects \cite{User_Interface_Modeling_in_UMLi}. Figure \ref{content/how_user_interfaces_are_built/umli_login.png} shows our login window example modelled using UMLi.
\image{6cm}{content/how_user_interfaces_are_built/umli_login.png}{Login window modelled in UMLi}

With this notation you can specify inputs, outputs and actions in a way that classic UML notation doesn’t support. Tasks can also be specified in UMLi, but without any extension to UML. Tasks can be modeled using Use Cases and Activity Diagrams which are part of the UML specification.

Concur task trees -> Falta falar disto!