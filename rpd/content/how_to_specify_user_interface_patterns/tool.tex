\subsection{Proposed tool}
Earlyer in this section we studied a couple of languages with the potential to become standard in user interface patterns specification. UsiXML is a more concise language while UsiPXML can store more information in a structured way by merging UsiXML with PLML which is a language for descriptive patterns. In conclusion UsiPXML seems to be more suited to describe patterns but UsiXML is more generic and thus as more potential to become a standard in the software engineering community so it's probably the best option on the table.

The tool proposed by this paper should have the following list of key features:
\begin{itemize}
\item Read and interpret patterns specified in UsiXML models. These models have to specify at least the abstract user interface model but the a task and context models should also be used if present. There should also be present a domain model and a mapping model to bind with the source code.
\item Read and interpret source code of one or more OOP language with annotations in a separate XML file. These annotations should contain information to help the binding process with the pattern and provide additional information such as regarding to validations.
\item Generate a concrete user interface resulting from a transformation of the pattern, taking into account the information gathered from the source code, in one or more programming languages and frameworks.
\end{itemize}