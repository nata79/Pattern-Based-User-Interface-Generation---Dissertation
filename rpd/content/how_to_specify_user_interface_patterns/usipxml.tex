\subsection{UsiPXML}
UsiPXML results from the fusion of two languages, PLML \cite{Pattern_Language_Markup_Language} and UsiXML \cite{Different_kinds_of_pattern_support_for_interactive_systems}. UsiXML was already studied in the last subsection so in the present subsection we’ll focus on the other components of UsiPXML.
\image{8cm}{content/how_to_specify_user_interface_patterns/usipxml.png}{Structure of UsiPXML.}
PLML provides the contextual information of a pattern in UsiPXML. The main goal of PLML is to bring structure and consistency to the way patterns are described. PLML is a natural language-based so it implements descriptive patterns. 

It wouldn't make sense to use PLML alone with the objective of creating generative patterns but using it along with UsiXML seems a good idea because these two languages complement each other in this context. On section \ref{section:Patterns_in_software_engineering} was stated that a pattern was composed by a name, a problem, a solution and a list of consequences. Using UsiPXML the solution can be described in UsiXML while the other components fit in the structure of PLML.
