\section{Conclusions}
User interface development is one of the most important phases in software development but still's very hard for developers to manage this process efficiently. The data from section \ref{section:How_user_interfaces_are_built} show's that there is a need for better tools and methodologies to build user interfaces.

Patterns are very important for engineers, in section \ref{section:Patterns_in_software_engineering} we studied how are patterns used, documented and stored. We divided patterns in two categories, descriptive patterns and generative patterns. Although the ones that are more interesting for this work are generative pattens, there is more abundance of descriptive patters for user interfaces. The main reason for this is the lack of a standard language for specifying user interfaces, like UML for general software.

On section \ref{section:How_to_specify_user_interface_patterns} we studied a couple a languages with the potential to become standard in user interface patterns specification. UsiXML is a more concise language while UsiPXML can store more information in a structured way by merging UsiXML with PLML which is a language for descriptive patterns. In conclusion UsiPXML seems to be more suited to describe patterns but UsiXML is more generic and thus as more potential to become a standard in the software engineering community so it's probably the best option on the table.

The future work for this project will be a more profound study of the UsiXML specification in order to develop a tool capable of reading a pattern described in UsiXML, link it with existing source code (business layer) and generate a concrete user interface.