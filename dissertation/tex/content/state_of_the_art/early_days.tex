\section{Early days}
\label{section:early_days}

Research in model-based user interface development comes from the 1980s\cite{SzekelyRetrospective}. Their roots come from user interface management systems (UIMS)\cite{MyersUIMS}. These tools seeked to provide an alternative paradigm for constructing interfaces. Rather than using a toolkit library, developers would write a specification in a specialized, high-level specification language. This specification would be automatically translated into an executable program, or interpreted at run-time to generate the appropriate interface.

Through the 1980s and 1990s specification languages became more sophisticated, supporting richer and more detailed representations that allowed systems to generate more sophisticated interfaces.

Since that time there were essentially two approaches to model driven development of user interfaces. Some tried to minimize the work of developers and generate most of the model from just a domain model like in Janus\cite{janus} or a task model as in Trident\cite{trident1, trident2}. The second approach was to give more power to the developer letting him produce all or most of the work like in Its\cite{ITS}.

This section includes a more in depth study of one project from each category. On section \ref{subsection:janus} the JANUS project and on section \ref{subsection:ITS} ITS.

\subsection{Janus}
\label{subsection:janus}
\input{content/state_of_the_art/early_days/its}