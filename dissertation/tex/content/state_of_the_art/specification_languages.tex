\section{Specification languages}
\label{section:specification_languages}

In \cite{idealxml2} was stated that one of the most important challenges to overcome in model driven development of user interfaces is the creation of a specification language that would be massively adopted and became a common ground between developers like UML is for software architecture.

Over the years many languages were developed to try and overcome this obstacle. In \cite{mecano} were enumerated some of the problems found with that time user interface models:
\begin{itemize}
\item \textbf{Partial models}, most models deal only with a portion of the spectrum of interface characteristics. Some emphasize domain, others emphasize tasks, some others emphasize presentation guidelines and so on.

\item \textbf{Insufficient underlying model}, several model-based systems use modelling paradigms proven successful in other applications areas, but that come up short for interface development. These underlying models typically result in partial interface models of restricted expressiveness.

\item \textbf{System-dependent models}, many interface models are non-declarative and are embedded implicitly into their associated model-based systems, sometimes at code level. These generic models are tied to the interface generation schema of their system, and are therefore unusable in any other environment.

\item \textbf{Inflexible models}, experience with model-based systems suggests that interface developers often wish to change, modify, or expand the interface model associated with a particular model-based environment. However, model-based systems do not offer facilities to such modifications, nor the interface models in question are defined in a way that modifications can be easily accomplished. Thus the inclusion of an open meta-model like in UML could be an important factor of success.
\end{itemize}

Next, two recent specification languages for user interfaces that try to overcome these problems will be presented. In section \ref{subsection:umli} will be presented UMLi, an extension to UML to support the modelling of user interfaces. In section \ref{subsection:usixml} will be presented UsiXML, a language with potential to become a standard in user interface specification.

\subsection{UMLi}
\label{subsection:umli}

Although user interfaces represent an essential part of software systems, UML seems to have been developed with little attention to specific details of user interface models. It's possible to use UML to model important aspects of user interfaces but these models usually get widely unnatural.

UMLi\cite{User_Interface_Modeling_in_UMLi} doesn't try to replace UML entirely. The UMLi meta-model fully integrates with UML this makes possible to integrate UMLi models with other UML models.

It is possible to model abstract and concrete interfaces using class models in UML. However class models don't provide an intuitive representation of the interface. UMLi provides an alternative
diagram notation for describing abstract interaction objects.

\image{6cm}{content/state_of_the_art/specification_languages/umli_login.png}{Login window modelled in UMLi}

Figure \ref{content/state_of_the_art/specification_languages/umli_login.png} shows an abstract user interface for a login windows modelled in UMLi. The upper container has four entities, \textit{username} and \textit{password} represent input controls while \textit{UsernameParam} and \textit{PasswordParam} are bindings to where the content of inputs will be stored. On the lower container are represented the actions of the window.

UMLi's user interface diagram consists of six constructors:
\begin{itemize}
\item \textbf{FreeContainers} rendered as dashed cubes. A \textit{FreeContainer} is a top-level interaction class that no other interaction class can contain.

\item \textbf{Containers} rendered as dashed cylinders. A Container is a mechanism that groups interaction classes other than \textit{FreeContainers}.

\item \textbf{Inputters} rendered as downward triangles. An \textit{Inputter} receives information from users.

\item \textbf{Editors} rendered as rhombi. An Editor facilitates the two-way exchange of information.

\item \textbf{Displayers} rendered as upward triangles. A Displayer sends information to users.

\item \textbf{ActionInvokers} rendered as right-pointing arrows. An \textit{ActionInvoker} receives direct instructions from users.
\end{itemize}

Tasks are usually represented in a tree notation in which leaf nodes are primitive tasks and non-leaf nodes group and describe relationships between their children nodes. There is a set of three essential features present in most task modelling languages:
\begin{itemize}
\item \textbf{Hierarchical decomposition}, high-level tasks systematically decompose into less abstract tasks.

\item \textbf{Temporal relationships}, the order in which a composite task's children are carried out depends on the parent's temporal relation.

\item \textbf{Primitive tasks}, the lowest-level nodes described in the task model are primitive tasks. An action task, for example, corresponds to an activity the application carries out. An interaction task involves some degree of human-computer interaction.
\end{itemize}

Use cases and activities in UMLi represent the notion of task with a set of features that include all the elementary ones mentioned above.

Using use cases and their scenarios, it's possible to elicit user interface functionalities required to let users achieve their goals. Possible ways to perform actions that support the functionalities elicited using use cases can be identified using activities. Therefore, mapping use cases into top level activities can help describe a set of interface functionalities similar to that described by
task models in other specification languages.

Use-case diagrams in UMLi are UML use-case diagrams. Activity diagrams in UMLi, however, extend activity diagrams in UML. UMLi provides a notation for a set of macros for activity diagrams that can be used to model behaviour categories usually observed in user interfaces: optional, order independent, and repeatable behaviours.

Using these macro notations, activity diagrams in UMLi can cope better with the tendency that activity diagrams have to become complex even when modelling the behaviour of simple user interfaces.

In order to represent relationships between models in UMLi object flows are used in activity diagrams to describe how to use class instances to perform actions in action states. By using object flows, it's possible to incorporate the notion of state into activity diagrams that are primarily used for modelling behaviour. In UMLi, it's also possible to use object flows to describe how to use interaction class instances. However, object flow states—rendered as dashed arrows connecting objects to action states—have specific semantics when associating interaction objects to activities and action states. UMLi specifies categories of object flow states specific to interaction objects:
\begin{itemize}
\item The \textbf{interacts} object flows relate primitive interaction objects to action states, which are primitive activities. They indicate that associated action states are responsible for interactions in which users invoke object operations or visualize the results of object operations.

\item The \textbf{presents} object flows relate \textit{FreeContainers} to activities and specify
that the associated \textit{FreeContainers} should be visible while the activities are active.

\item The \textbf{confirms} object flows relate \textit{ActionInvokers} to selection states and specify that selection states have finished normally.

\item The \textbf{cancels} object flows relate \textit{ActionInvokers} to composite activities or selection states and specify that activities or selection states have not finished normally and that the application flow of control should be rerouted to a previous state.

\item The \textbf{activates} object flows relate \textit{ActionInvokers} to other activities, thereby triggering the associated activities that start when an event occurs.
\end{itemize}

In \cite{User_Interface_Modeling_in_UMLi} is mentioned a case study specified both in UML and UMLi. A set of metrics were applied to each specification. The results of these metrics show that constructing and maintaining interactive system models should be simpler and easier in UMLi than in UML.
\chapter{UsiXML}

In section \ref{subsection:usixml} UsiXML was referred as part of the state of the art. In this chapter this subject will be studied in more depth. The focus of this chapter will not be just UsiXML as a specification language but the whole project behind it.

UsiXML's last version was released in February 14th, 2007. In April of 2009 was founded a consortium with the objective of lowering  the total application costs and development time by adding versatile context driven capabilities to UsiXML that would bring it far beyond the state of the art, up to the achievement of its standardisation.

The UsiXML consortium started with three main goals that would guide the entire project. The first goal is \textit{the UsiXML ``$\mu7$'' concept elicitation and promotion}. The $\mu7$ concept means that it should be possible to specify  a user interface for an interactive application that should support multi-device, multi-platform, multi-user, multi-linguality/culturality, multi-organisation, multicontext, or multi-modality capabilities.

The second goal is the \textit{development of the UsiXML language and the model-driven method}. The UsiXML language should guarantee interoperability, reusability, and maintainability of interactive applications developed. For this reason UsiXML is an open XML-compliant standard User Interface Description Language (UIDL). There should be models to cover every $\mu7$ aspects. Finally, UsiXML should define a flexible methodological framework that accommodates various development paths as found in organisations and that can be tailored to their specific needs.

The third goal is to \textit{set up development tools and demonstration of the validity on applications}. There should be a suite of software tools that support the methodological framework defined in goal 2 that can be later integrated or connected to available software environments. A UsiXML service will be defined and developed once so that this service can be deployed many times, especially for all environments requiring them to reduce the total cost of development. UsiXML will provide developers with various knowledge bases containing usability and accessibility guidelines that can be semi-automatically verified on any UsiXML-produced UI so as to guarantee a certain level of quality.

Section \ref{section:usixml_semantics} provides an in depth study of the UsiXML semantics. In section \ref{section:usixml_method} refers to the UsiXML engineering method for developing user interfaces.

\section{UsiXML semantics}
\label{section:usixml_semantics}

The core component of a user interface specified in UsiXML
consists of a uiModel, which is itself decomposed into several models. Not all models should be included. Only those models which are required for the particular user interface are included in a UsiXML file. Figure \ref{content/usixml/uml_spec.png} shows the components of UsiXML in an UML class diagram.

\image{\textwidth}{content/usixml/uml_spec.png}{UsiXML specification as an UML class diagram.}

The uiModel is the topmost superclass containing common features shared by all component models of a user interface. A uiModel may consist of a list of component models in any order and any number:
\begin{itemize}
\item \textbf{Transformation model}: contains a set of rules in order to enable a transformation of one specification to another.
\item \textbf{Domain model}: describes the classes of the objects manipulated by the users while interacting with the system.
\item \textbf{Task model}: describes the interactive task as viewed by the user interacting with the system. The task model is expressed according to the CTT specification \cite{ConcurTaskTrees_A_Diagrammatic_Notation_for_Specifying_Task_Models}.
\item \textbf{Abstract user interface model}: represents the view and behavior of the domain concepts and functions in platform independent way.
\item \textbf{Concrete user interface model}: represents a concretization of the abstract user interface model.
\item \textbf{Mapping model}: contains a series of related mappings between models or elements of models.
\item \textbf{Context model}: describes the three aspects of a context of use, which is a user carrying out an interactive task using a specific computing platform in a given surrounding environment.
\item \textbf{Resource model}: contains definitions of resources attached to abstract or concrete interaction objects.
\end{itemize}
The uiModel is also composed by a creation date, a schema version, a set of authors, a set of versions and a set of comments. A complete reference to UsiXML semantics can be found in \cite{UsiXML_USer_Interface_eXtensible_Markup_Language}.
\section{UsiXML method}
\label{section:usixml_method}

The MDE approach allows developing the UsiXML UI by transforming progressively the UsiXML models to obtain specifications that are detailed and precise enough to be rendered or transformed into code.

\subsection{Spem 2.0}
\label{subsection:spem2}

SPEM is an OMG standard dedicated to software method modelling. The goal of SPEM is to propose minimal elements necessary to define any software and systems development method, without adding specific features to address particular domains. As a result, this meta-model supports a large range of development methods of different styles, levels of formalism, and life-cycle models.

SPEM is a UML profile, meaning that SPEM reuses UML whenever possible. Consequently, SPEM uses UML for various software model concepts presentations.

The current version of SPEM is 2.0 and it was completely reformulated from the previous one in order to separate the operational aspect of a method from the temporal aspect of a method. As shown in figure \ref{content/usixml/method/spem_base.png}, the SPEM 2.0 meta-model uses seven main meta-model packages:
\begin{itemize}
\item \textbf{Method Content} package describes the static aspect of a method; 
\item \textbf{Process Structure} and \textbf{Process Behaviour} packages describe the dynamic aspect of a method, \textbf{Process With Methods} package describes the link between these two aspects; 
\item \textbf{Core package} provides the common classes that are used in the different packages; \item \textbf{Method Plug-in} package describes the configuration of a method and Managed Content package describes the documentation of a method.
\end{itemize}

\image{12cm}{content/usixml/method/spem_base.png}{Structure of the SPEM 2.0 meta-model.}

In the following subsections we'll go into some detail for each package in order to get a better understanding of the SPEM 2.0 method meta-model.

\subsubsection{Core package}

The Core meta-model package contains abstract generalization classes that are specialized in the other meta-model packages. These abstract generalization classes are used to define
common properties of their specialized classes. The main elements of the core package are:
\begin{itemize}
\item The \textbf{Kind class}, that expresses a refined vocabulary specific to a method;
\item \textbf{Work Definition} is an abstract generalization class that represents the work being performed by a specific role, or the work performed throughout a life-cycle.
\item \textbf{Work Definition Parameter} is an abstract generalization class that represents parameters for Work Definitions.
\item \textbf{Work Definition Performer} is an abstract generalization class that represents the relationship of a work performer (role) to a Work Definition.
\end{itemize}
\subsection{Spem4UsiXML}
\label{subsection:spem4usixml}

\subsection{Forward engineering method}
\label{subsection:forward_engineering_method}
