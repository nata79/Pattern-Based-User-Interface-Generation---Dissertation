\subsection{GADGET}
\label{subsection:GADGET}

Although optimization-based techniques appear to offer several potential advantages most programmers are intimidated or uncomfortable by the math required to program an optimization. Although optimization tool-kits are available (cite), they typically require substantial specialized knowledge because they have mostly been designed for physics simulations and other traditional optimization problems.

GADGET provides a set of abstractions for many optimization concepts along with a set of mechanisms to help programmers quickly create optimizations, including an efficient lazy evaluation framework, a powerful and configurable optimization structure, and a library of reusable components.
A programmer creating an optimization using the GADGET tool-kit needs to supply three essential components: an initializer, iterations and evaluations. The initializer creates the initial solutions to be optimized. This might be based on an existing algorithm, or done randomly. Iterations are responsible to transform one potential solution into another, typically using models that are at least partially random. Finally, evaluations are used to judge the different notions of goodness in a solution.

There’s a standard framework to abstract the concepts and constructs behind evaluations. GADGET allows programmers to focus on creating evaluations to measure criteria that are important to a problem. GADGET then combines these evaluations and uses them to between possible solutions to a problem. This process is divided into five stages. 

First the framework presents each evaluation with the current potential solution, which is called the prior solution. Each evaluation returns an array of double values representing its interpretation of the prior solution. This collection of arrays of double values is called the prior result. 

On the second stage the framework uses an iteration object to modify the prior solution and create a new one, called the post solution.

Third, the framework presents the post solution to each evaluation. Each individual evaluation returns interpretations that are then combined to create a post result.

In the fourth step, the framework uses a method to compare the prior result with the post result. This is possible by requiring each evaluation to be capable of comparing two arrays of double values that it has created and providing a double value in the range of -1 to 1, where -1 indicates the evaluation has a strong preference for the prior solution, 0 indicates that the evaluation is indifferent and 1 indicates that the evaluation has a strong preference for the post solution.

Finally, the result of this comparison indicates whether the framework should go with the post solution or revert to the prior solution. To choose between them, the values retrieved from the fourth step are multiplied by the weight associated with its respective evaluation and then summed. If the sum is greater than 0, the framework prefers the post solution, else it reverts to the prior solution. 