\section{User interface patterns}
\label{section:user_interface_patterns}

Patterns are widely used in every field of engineering. One of the earlier definitions of patterns can be found on \cite{A_Pattern_Language_Towns_Buildings_Construction}. Almost twenty years later patterns were brought to software engineering by \cite{Design_Patterns}.

Patterns bring many advantages, not only they make the development of a product less time consuming and thus less expensive but can also guarantee a higher level of quality because patterns are solutions that have been tested and used in other projects.

Particularly on user interfaces, these are very important features because building a good user interface is a very complex and time consuming process. On most software projects it takes about half of the time frame allocated to that project, so patterns can help to make this process more efficient. Also there's the problem of usability. This is one of the most important aspects of software projects but its still very difficult to build a user interface compliant with human computer interaction (HCI) rules. By using patterns this can be easily achieved if the patterns are already compliant with these rules.

Patterns are usually stored in catalogues. In \cite{Design_Patterns} a pattern is composed by the following fields:
\begin{itemize}
\item The \textbf{Pattern name} resumes the pattern in one or two words that we use to refer to named pattern.
\item The \textbf{Problem} describes in which situations the pattern should be applied.
\item The \textbf{Solution} describes how the pattern works, what elements it has and how they relate to each other.
\item The \textbf{Consequences} describe the side effects of using the pattern.
\end{itemize}
This is the specification used for software design patterns but it's also used in most user interface patterns catalogs.
In \cite{Generative_pattern-based_design_of_user_interfaces} documentation of patterns is divided in two categories, descriptive patterns and generative patterns. Descriptive patterns are meant to be interpreted by humans so they describe the solution in a generic way so that the pattern can be used in a wide range of contexts while generative patterns maximize \textit{expressivity} over \textit{genericity} thus, they can be used in more restricted range of contexts but the solution is specific enough to be interpreted by machines. 

Design patterns like the ones described in \cite{Design_Patterns} are generative patterns because their solution is specified in UML which is a formal language that can be easily interpreted by machines to perform transformations.

A list of catalogues for user interfaces can be found in \cite{The_Interaction_Design_Patterns_Page}. Most of these catalogues define their solutions with text and images because there isn't a reference language to specify user interfaces. Thus most of these patterns are descriptive patterns that can only be used by humans.

In order to take full advantage of patterns we need a way to document them. Generative patterns are the most useful in the context of this project but to use them we need to find a language to specify these patterns so that they can be interpreted by a machine to generate a concrete user interface. applications. In section \ref{subsection:IdealXML} is described IdealXML a tool for developing user interface models that takes advantages of patterns.

\subsection{IdealXML}
\label{subsection:IdealXML}

IdealXML\cite{IdealXml_An_Interaction_Design_Tool, idealxml2} is an experience-based environment for user interface design. Experience is the accumulation of knowledge or skills that result from direct participation in events or activities. Developers have a strong tendency to towards reusing designs that worked well for them in the past. Unfortunately, this design reuse is usually limited by personal experience, and there is usually few sharing of knowledge among developers.

IdealXML manipulates a pattern repository, where patterns are organized following a hierarchical structure. At the top, this structure has different models related with a MB-UIDE: domain, task, presentation and mapping, context and user models are left for future work. IdealXML is shipped with a predefined collection of patterns from a variety of sources. These patterns are the initial base of knowledge.

IdealXML is an MB-UIDE and designers can, using several graphical notations, specify domain models, task models, abstract presentation models and mapping models between them. Some of these models are stored in the pattern repository and new ones can always be added.

IdealXML also allows for the animation of a task model to generate a hi-fi prototype of the future user interface while still in the first development stages. This is achieved by using CTT, UsiXML and a set of heuristics to transform the task model specification into an abstract UI.

Prototyping consists in the creation of a preliminary version of the future UI (prototype) so that the user and the experts can find possible problems in the design of the UI, both from the functional and from the usability points of view. Prototyping techniques fall into two main categories:
\begin{itemize}
\item \textbf{Lo-fi:} this family of techniques is mostly used in requirements analysis stage to validate the requirements with the user in user-centred approaches.

\item \textbf{Hi-fi:} they are aimed at the creation of preliminarily versions of the UI with an acceptable degree of quality. This kind of techniques produces a UI prototype which is closer to final future one.
\end{itemize}

Abstract prototyping was devised because it was found that the sooner developers started drawing realistic pictures or positioning real widgets, the longer it took them to converge on a good design.

As it was mentioned above, IdealXML uses a set of heuristics to transform the task model into an abstract interface model:
\begin{itemize}
\item Each cluster of interrelated task cases becomes an interaction space in the navigation map, so an abstract task is a container.
\item A container also can be an interaction task or an application task in a hierarchical task decomposition.
\item A component rises when an interaction or application task is found in a hierarchical task decomposition.
\item A component can have several facets (input, output, control and navigation). These facets allow the user to interact with the system.
\end{itemize}

The animation of the abstract user interface that resulted from the designed task model is grounded in the identification of the enabled task set (ETS). Having identified the ETC for a task model, the next step is to identify the effects of performing each task in each ETS. The result of this analysis is  a state and transitions occur when tasks are performed. In IdealXML's proposal, the task model specification is split into states. Each state is a set of interrelated tasks, including temporal relationships between those tasks, usually connected to an essential use case.