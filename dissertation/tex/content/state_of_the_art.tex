\chapter{State of the art}
In this chapter is described the state art regarding model driven development of user interfaces. It will be presented several techniques to develop user interfaces using model driven principles. There is significant ongoing research in this field since the late 1980s. This chapter makes a short summary of that research. Some works will be examined in more depth for the sake of example. 

Model driven development defining characteristic is that software development's primary focus and products are models rather than computer programs. The major advantage of this is that we express models using concepts that are much less bound to the underlying implementation technology and are much closer to the problem domain relative to most popular programming languages \cite{The_Pragmatics_of_Model-Driven_Development}.

Models are easier to maintain than the code itself and, most importantly, they're platform independent. This means that the same model can be used to generate code that runs on a desktop environment, a web environment or even a mobile environment. This makes a very important feature for user interfaces because modern applications are becoming more and more ubiquitous and it's highly complex and time consuming to build a user interface for every supported platform.

There are different kinds of models and several techniques to model user interfaces. Some tools like Janus\cite{janus} only require a domain model to generate a concrete user interface while others like Trident\cite{trident1, trident2} and Adept\cite{adept1} also use a task model. All these approaches automate part of the development process. Other approaches like ITS\cite{ITS} require more work from the developers because it doesn't generate any model.

Some more recent work have been using different techniques to generate user interfaces. Gadget\cite{gadget} or Supple\cite{supple} have been using optimization techniques to generate better concrete user interfaces

Patterns are widely used in every field of engineering. One of the earlier definitions of patterns can be found on \cite{A_Pattern_Language_Towns_Buildings_Construction}. Almost twenty years later patterns were brought to software engineering by \cite{Design_Patterns}.

Patterns bring many advantages, not only they make the development of a product less time consuming and thus less expensive but can also guarantee a higher level of quality because patterns are solutions that have been tested and used in other projects. IdealXML\cite{IdealXml_An_Interaction_Design_Tool, idealxml2} is a tool that helps developers to take advantage of patterns while developing user interface models.

Although there have been a lot of academic work surrounding model driven development of user interfaces, this work is struggling to be adopted by the industry. There are some theories for what is missing, in \cite{molina} is said that better tool support is needed and in \cite{IdealXml_An_Interaction_Design_Tool} is stated that a common language for representing user interface models is the next step.

UML \cite{The_Unified_Modeling_Language_Reference_Manual} is the industry standard for software modelling but, unfortunately, is not fit to model user interfaces. Fortunately, the software engineering community has developed some new modelling languages in the past few years to overcome this problem. UMLi \cite{User_Interface_Modeling_in_UMLi} is an extension to UML that provides an alternative diagram notation for describing abstract interaction objects. ConcurTaskTrees (CTT) \cite{ConcurTaskTrees_A_Diagrammatic_Notation_for_Specifying_Task_Models} aims at task modelling by dividing the task model is built in three essential parts:
\begin{itemize}
\item First a hierarchical logical decomposition of the tasks represented by a tree-like structure;
\item Then an identification of the temporal relationships among tasks at the same level;
\item And finally an identification of the objects associated with each task and of the actions which allow them to communicate with each other.
\end{itemize} 

UsiXML is a user interface description language aimed at expressing user interfaces built with various modalities of interaction and independently of them. UsiXML is XML compliant to enable flexible exchange of information and powerful communication between models and tools used in user interface engineering \cite{UsiXML_USer_Interface_eXtensible_Markup_Language}. One of the great advantages of UsiXML is platform independence providing a multi-path development of user interfaces \cite{UsiXML_a_Language_Supporting_Multi-Path_Development_of_User_Interfaces}. UsiXML characteristics and features give it the potential to become a standard for modelling user interfaces like UML is for software architecture.

There is a lot of work regarding model driven development for user interfaces and the idea that models can simplify the development process is becoming more consensual. Section \ref{section:early_days} focus on projects from the beginnings of model driven development of user interfaces, namely Janus on section \ref{subsection:janus} and Its, section \ref{subsection:ITS}.

Section \ref{section:optimization_based_generation_of_interfaces} will describe optimization based techniques and the Gadget and Supple projects will be examined with more depth.

Section \ref{section:user_interface_patterns} refers to the usage of patterns in user interface design . This section also includes an analysis of IdealXML, a knowledge based tool for designing user interfaces.

Section \ref{section:specification_languages} is about specification languages. UMLi and UsiXML will be analysed in details in sections \ref{subsection:umli} and \ref{subsection:usixml} respectively.

\section{Early days}
\label{section:early_days}

Research in model-based user interface development comes from the 1980s\cite{SzekelyRetrospective}. Their roots come from user interface management systems (UIMS)\cite{MyersUIMS}. These tools seeked to provide an alternative paradigm for constructing interfaces. Rather than using a toolkit library, developers would write a specification in a specialized, high-level specification language. This specification would be automatically translated into an executable program, or interpreted at run-time to generate the appropriate interface.

Through the 1980s and 1990s specification languages became more sophisticated, supporting richer and more detailed representations that allowed systems to generate more sophisticated interfaces.

Since that time there were essentially two approaches to model driven development of user interfaces. Some tried to minimize the work of developers and generate most of the model from just a domain model like in Janus\cite{janus} or a task model as in Trident\cite{trident1, trident2}. The second approach was to give more power to the developer letting him produce all or most of the work like in Its\cite{ITS}.

This section includes a more in depth study of one project from each category. On section \ref{subsection:janus} the JANUS project and on section \ref{subsection:ITS} ITS.

\subsection{Janus}
\label{subsection:janus}
\input{content/state_of_the_art/early_days/its}
\section{Optimization-based generation of interfaces}
\label{section:optimization_based_generation_of_interfaces}

Recent work is beginning to reveal that numerical optimization can play a role on modern approaches for generating interfaces and displays. GADGET\cite{gadget} and Supple\cite{supple} are two examples of this tendency.

The first is a framework that aims to provide developers with none or few knowledge on numerical optimization a set of tools that allows them to generate user interfaces through optimization methods.

The second is a tool that aims the generation of personalized user interfaces on run time. The main motivation behind Supple is that current user interfaces are developed with only a limited set of user abilities in mind leaving people with special needs with difficulties to interact with their applications.

\subsection{Gadget}
\label{subsection:GADGET}

Although optimization-based techniques appear to offer several potential advantages most programmers are intimidated or uncomfortable by the math required to program an optimization. Although optimization tool-kits are available, they typically require substantial specialized knowledge because they have mostly been designed for physics simulations and other traditional optimization problems.

Gadget provides a set of abstractions for many optimization concepts along with a set of mechanisms to help programmers quickly create optimizations, including an efficient lazy evaluation framework, a powerful and configurable optimization structure, and a library of reusable components.
A programmer creating an optimization using the Gadget tool-kit needs to supply three essential components: an initializer, iterations and evaluations. The initializer creates the initial solutions to be optimized. This might be based on an existing algorithm, or done randomly. Iterations are responsible to transform one potential solution into another, typically using models that are at least partially random. Finally, evaluations are used to judge the different notions of goodness in a solution.

There’s a standard framework to abstract the concepts and constructs behind evaluations. Gadget allows programmers to focus on creating evaluations to measure criteria that are important to the problem. Gadget then combines these evaluations and uses them to choose between a set of possible solutions. This process is divided into five stages. 

First the framework presents each evaluation with the current potential solution, which is called the prior solution. Each evaluation returns an array of double values representing its interpretation of the prior solution. This collection of arrays of double values is called the prior result. 

On the second stage the framework uses an iteration object to modify the prior solution and create a new one, called the post solution.

Third, the framework presents the post solution to each evaluation. Each individual evaluation returns interpretations that are then combined to create a post result.

In the fourth step, the framework uses a method to compare the prior result with the post result. This is possible by requiring each evaluation to be capable of comparing two arrays of double values that it has created and providing a double value in the range of $-1$ to $1$, where $-1$ indicates the evaluation has a strong preference for the prior solution, $0$ indicates that the evaluation is indifferent and 1 indicates that the evaluation has a strong preference for the post solution.

Finally, the result of this comparison indicates whether the framework should go with the post solution or revert to the prior solution. To choose between them, the values retrieved from the fourth step are multiplied by the weight associated with its respective evaluation and then summed. If the sum is greater than 0, the framework prefers the post solution, else it reverts to the prior solution. 
\subsection{Supple}
\label{subsection:Supple}

Most of today's software interfaces are designed with the assumption that they are going to be used by an able-bodied person, who is using a typical set of input and output devices, who has typical perceptual and cognitive abilities, and who is sitting in a stable, warm environment. Any deviation from this pattern requires a new design. In \cite{supple} is argued that automatic personalized generation of interfaces is a feasible and scalable solution for this challenge. Supple can automatically generate interfaces adapted to a persons devices, tasks, preferences and abilities at run-time. It is not intended to replace user interface designers, instead it offers an alternative user interface for those people whose devices, tasks, preferences and abilities are not sufficiently addressed by the original designs.

Support for users with special needs is often forgotten by interface designers. When this problem is addressed there are three popular patters that are usually followed, manual redesign of the interface, limited customization support, or by supplying an external assistive technology. The first approach is clearly not scalable, new devices constantly enter the market, and people's abilities or preferences both differ greatly and often cannot be anticipated in advance. Second, today's customization approaches typically only support changes to the organization of tool bars and menus and cosmetic changes to other parts of the interface. Furthermore, even when given the opportunity, most people do not customize their applications. Finally, assistive technologies, while they often enable computer access for people who would otherwise not have it, also have limitations. They're impractical for users with temporal impairments, they do not adapt to people whose abilities change over time and they're often abandoned by people who need them because of factors like cost, complexity, configuration, and the need for ongoing maintenance.

In contrast with this approach, Supple generates personalized interfaces to suit the particular contexts of individual users. In order to be able to generate these personalized interfaces, Supple makes three important contributions:
\begin{itemize}
\item Defines an interface generation as a discrete constrained optimization problem and solve it with a branch-and-bound algorithm using constraint propagation. This general approach allows the Supple system to automatically generate ``optimal'' user interfaces given a declarative description of an interface, device characteristics, available widgets, and a user and device specific cost function.

\item Two types of cost functions were developed to guide the optimization process. The first is factored in a manner that enables preference-based personalization as well as fast computation, allowing Supple to generate user interfaces in under 1 second in most cases. The second explicitly models a person's ability to control the pointer, allowing Supple to generate user interfaces adapted to unusual interaction techniques or abilities, such as an input jittery eye tracker or a user's limited range of motion due to a motor impairment.

\item Supple also supports two approaches for dynamic personalization of generated interfaces: an automatic system driven adaptation to the current task, and a user driven customization.
\end{itemize}

Like other automatic user interface generation systems, Supple relies on an interface specification ($I$). Additionally, Supple also uses an explicit device model ($D$) to describe the capabilities and limitations of the platform for which the interface is to be generated. Finally, in order to reflect individual differences among usage patterns, Supple additionally includes a usage model, represented in terms of user traces ($T$).

Supple adopts a functional representation of user interfaces that is, one that says what functionality the interface should expose to the user instead of how to present those features. Like a number of previous systems, Supple represents basic functionality in terms of types of data that need to be exchanged between the application and the user. Semantic groupings of basic elements are expressed through container types which also serve as reusable abstractions. Supple chooses not to adopt a task-oriented approach for two reasons. First, because task-oriented descriptions are typically first compiled into a data-oriented functional description. Second, task-oriented languages are particularly useful for capturing task-oriented processes such as store checkout or making a hotel reservation. Most direct manipulation systems, however, support a broad range of possible tasks and make simultaneously available numerous reversible actions. Such interfaces would not benefit significantly from a task-oriented representation.

Formally an interface is defined to be $I \equiv \langle S_{f}, C_{I} \rangle $, where $S_{f}$ is a tree of interface elements, and $C_{I}$ is a set of interface constraints specified either by the designer at design time, or by the user at run time through Supple's customization mechanism. The interface elements included in the functional specification correspond to units of information that need to be conveyed via the interface between the user and the controlled application. The interface constraints can, in principle, constrain any aspect of interface presentation. In practice, it relies on the following three classes of constraints:
\begin{itemize}
\item \textbf{equality constraints}, which allow multiple instances of the same type to be rendered identically;

\item \textbf{constraints limiting the set of presentation options for an element}, which allow the user to customize the interface and choose from a set of available widgets to represent an element;

\item \textbf{interdependence constraints}, for example, not letting a check-box to be rendered alone inside of a tab pane.
\end{itemize}

The elements in the functional specification are defined in terms of their types. There are several classes of types:
\begin{itemize}
\item \textbf{Primitive types} include the common basic data types such as integers, floats, strings and booleans.

\item \textbf{Container types}, formally represented as $\lbrace \tau_{1}, \tau_{2}, …, \tau_{n} \rbrace$, are used to create groups of simpler elements, $\tau_{i}$.

\item \textbf{Constrained types}, $ \langle \tau, C_{\tau} \rangle $, denotes a constrained type, where $\tau$ is any primitive or container type and $C_{\tau}$ is a set of constraints over the values of this type. The constraints can be of any type, but typically they are expressed as an enumeration of allowed values or as a range.

\item \textbf{Vectors elements}  $ ( \langle \tau, C_{\tau} \rangle ) $ denote an ordered sequence of zero or more values of type $\tau$ and are used to support multiple selection. Like in the constrained types, the constraints $C_{\tau}$ define the set of values of type $\tau$ that can be selected.

\item \textbf{Actions} are denoted with a functional type signature, $\tau_{1} \vdash \tau_{2}$, where $\tau_{1}$ stands for the type of the object containing parameters of the action and $\tau_{2}$ describes the return type, that is, the interface component that is to be displayed after the typical execution of the action. Unlike the other types which are used to represent an application's state, the action type is used to invoke application's methods.
\end{itemize}

A display-based device, in Supple, is modelled as a tuple, $D \equiv \langle W, C_{D} \rangle $, where $W$ is the set of available user interface widgets on that device and $C_{D}$ denotes a set of device-specific constraints. Widgets are objects that turn elements from the functional specification into components of a rendered interface. Like the interface constraints, the device-specific constraints in $C_{D}$ are simply functions that map a full or partial set of element-widget assignments to either $true$ or $false$.

Most people use only a small subset of functions available in any application, and different users use different subsets. To adapt to a person's tasks and long-term usage patterns, the user interface should be rendered such that important functionalism is easy to manipulate and to navigate to. Instead of relying on explicit annotations by the designer or the user, Supple relies on usage traces, which can correspond either to actual or anticipated usage. Usage traces provide not just interaction frequency for primitive widgets, but also frequencies of transitions among different interface elements. In the context of the optimization framework, traces offer the possibility of computing \textit{expected} cost with respect to anticipated use.

A usage trace $T$, is a set of \textit{trails} where the term trail refers to ``coherent'' sequences of elements manipulated by the user. Supple assumes that a trail ends when the interface is closed or otherwise reset. A trail $T$ is a sequence of events $u_{i}$, each of which is a tuple $ \langle e_{i}, V_{old_{i}}, V_{new_{i}} \rangle $. Here $e_{i}$ is the interface element manipulated and $V_{old_{i}}$ and $V_{new_{i}}$ refer to the old and new values this element assumed.

The goal of all these specifications is to render each interface element with a concrete widget. Thus a \textit{legal} rendering of a functional specification $S_{f}$ is defined to be a mapping $R: S_{f} \vdash W$ which satisfies the interface and device constraints in $C_{I}$ and $C_{D}$. Of course there are many \textit{legal} renderings. Therefore, in order to find the best one, Supple relies on a cost function $\$: R \vdash \Re \geq 0$, which provides quantitative metric of the user interface quality.

Supple is an optimization-based tool to generate user interfaces. Unlike GADGET it is not a simple framework to be used by programmers, it implements all the optimization related logic. Developers are only obliged to supply the specification. Unlike most of the recent tools in this field, Supple only requires the abstract user interface model and a set of constraints, it doesn't use tasks. The main advantage regarding other solutions is that constraint set is dynamic and depends both on the user and the device. This allows Supple to generate good user interfaces to every user even if the designer doesn't have much information about the users preferences.
\section{User interface patterns}
\label{section:user_interface_patterns}

Patterns are widely used in every field of engineering. One of the earlier definitions of patterns can be found on \cite{A_Pattern_Language_Towns_Buildings_Construction}. Almost twenty years later patterns were brought to software engineering by \cite{Design_Patterns}.

Patterns bring many advantages, not only they make the development of a product less time consuming and thus less expensive but can also guarantee a higher level of quality because patterns are solutions that have been tested and used in other projects.

Particularly on user interfaces, these are very important features because building a good user interface is a very complex and time consuming process. On most software projects it takes about half of the time frame allocated to that project, so patterns can help to make this process more efficient. Also there's the problem of usability. This is one of the most important aspects of software projects but its still very difficult to build a user interface compliant with human computer interaction (HCI) rules. By using patterns this can be easily achieved if the patterns are already compliant with these rules.

Patterns are usually stored in catalogues. In \cite{Design_Patterns} a pattern is composed by the following fields:
\begin{itemize}
\item The \textbf{Pattern name} resumes the pattern in one or two words that we use to refer to named pattern.
\item The \textbf{Problem} describes in which situations the pattern should be applied.
\item The \textbf{Solution} describes how the pattern works, what elements it has and how they relate to each other.
\item The \textbf{Consequences} describe the side effects of using the pattern.
\end{itemize}
This is the specification used for software design patterns but it's also used in most user interface patterns catalogs.
In \cite{Generative_pattern-based_design_of_user_interfaces} documentation of patterns is divided in two categories, descriptive patterns and generative patterns. Descriptive patterns are meant to be interpreted by humans so they describe the solution in a generic way so that the pattern can be used in a wide range of contexts while generative patterns maximize \textit{expressivity} over \textit{genericity} thus, they can be used in more restricted range of contexts but the solution is specific enough to be interpreted by machines. 

Design patterns like the ones described in \cite{Design_Patterns} are generative patterns because their solution is specified in UML which is a formal language that can be easily interpreted by machines to perform transformations.

A list of catalogues for user interfaces can be found in \cite{The_Interaction_Design_Patterns_Page}. Most of these catalogues define their solutions with text and images because there isn't a reference language to specify user interfaces. Thus most of these patterns are descriptive patterns that can only be used by humans.

In order to take full advantage of patterns we need a way to document them. Generative patterns are the most useful in the context of this project but to use them we need to find a language to specify these patterns so that they can be interpreted by a machine to generate a concrete user interface. applications. In section \ref{subsection:IdealXML} is described IdealXML a tool for developing user interface models that takes advantages of patterns.

\subsection{IdealXML}
\label{subsection:IdealXML}

IdealXML\cite{IdealXml_An_Interaction_Design_Tool, idealxml2} is an experience-based environment for user interface design. Experience is the accumulation of knowledge or skills that result from direct participation in events or activities. Developers have a strong tendency to towards reusing designs that worked well for them in the past. Unfortunately, this design reuse is usually limited by personal experience, and there is usually few sharing of knowledge among developers.

IdealXML manipulates a pattern repository, where patterns are organized following a hierarchical structure. At the top, this structure has different models related with a MB-UIDE: domain, task, presentation and mapping, context and user models are left for future work. IdealXML is shipped with a predefined collection of patterns from a variety of sources. These patterns are the initial base of knowledge.

IdealXML is an MB-UIDE and designers can, using several graphical notations, specify domain models, task models, abstract presentation models and mapping models between them. Some of these models are stored in the pattern repository and new ones can always be added.

IdealXML also allows for the animation of a task model to generate a hi-fi prototype of the future user interface while still in the first development stages. This is achieved by using CTT, UsiXML and a set of heuristics to transform the task model specification into an abstract UI.

Prototyping consists in the creation of a preliminary version of the future UI (prototype) so that the user and the experts can find possible problems in the design of the UI, both from the functional and from the usability points of view. Prototyping techniques fall into two main categories:
\begin{itemize}
\item \textbf{Lo-fi:} this family of techniques is mostly used in requirements analysis stage to validate the requirements with the user in user-centred approaches.

\item \textbf{Hi-fi:} they are aimed at the creation of preliminarily versions of the UI with an acceptable degree of quality. This kind of techniques produces a UI prototype which is closer to final future one.
\end{itemize}

Abstract prototyping was devised because it was found that the sooner developers started drawing realistic pictures or positioning real widgets, the longer it took them to converge on a good design.

As it was mentioned above, IdealXML uses a set of heuristics to transform the task model into an abstract interface model:
\begin{itemize}
\item Each cluster of interrelated task cases becomes an interaction space in the navigation map, so an abstract task is a container.
\item A container also can be an interaction task or an application task in a hierarchical task decomposition.
\item A component rises when an interaction or application task is found in a hierarchical task decomposition.
\item A component can have several facets (input, output, control and navigation). These facets allow the user to interact with the system.
\end{itemize}

The animation of the abstract user interface that resulted from the designed task model is grounded in the identification of the enabled task set (ETS). Having identified the ETC for a task model, the next step is to identify the effects of performing each task in each ETS. The result of this analysis is  a state and transitions occur when tasks are performed. In IdealXML's proposal, the task model specification is split into states. Each state is a set of interrelated tasks, including temporal relationships between those tasks, usually connected to an essential use case.
\section{Specification languages}
\label{section:specification_languages}

In \cite{idealxml2} was stated that one of the most important challenges to overcome in model driven development of user interfaces is the creation of a specification language that would be massively adopted and became a common ground between developers like UML is for software architecture.

Over the years many languages were developed to try and overcome this obstacle. In \cite{mecano} were enumerated some of the problems found with that time user interface models:
\begin{itemize}
\item \textbf{Partial models}, most models deal only with a portion of the spectrum of interface characteristics. Some emphasize domain, others emphasize tasks, some others emphasize presentation guidelines and so on.

\item \textbf{Insufficient underlying model}, several model-based systems use modelling paradigms proven successful in other applications areas, but that come up short for interface development. These underlying models typically result in partial interface models of restricted expressiveness.

\item \textbf{System-dependent models}, many interface models are non-declarative and are embedded implicitly into their associated model-based systems, sometimes at code level. These generic models are tied to the interface generation schema of their system, and are therefore unusable in any other environment.

\item \textbf{Inflexible models}, experience with model-based systems suggests that interface developers often wish to change, modify, or expand the interface model associated with a particular model-based environment. However, model-based systems do not offer facilities to such modifications, nor the interface models in question are defined in a way that modifications can be easily accomplished. Thus the inclusion of an open meta-model like in UML could be an important factor of success.
\end{itemize}

Next, two recent specification languages for user interfaces that try to overcome these problems will be presented. In section \ref{subsection:umli} will be presented UMLi, an extension to UML to support the modelling of user interfaces. In section \ref{subsection:usixml} will be presented UsiXML, a language with potential to become a standard in user interface specification.

\subsection{UMLi}
\label{subsection:umli}

Although user interfaces represent an essential part of software systems, UML seems to have been developed with little attention to specific details of user interface models. It's possible to use UML to model important aspects of user interfaces but these models usually get widely unnatural.

UMLi\cite{User_Interface_Modeling_in_UMLi} doesn't try to replace UML entirely. The UMLi meta-model fully integrates with UML this makes possible to integrate UMLi models with other UML models.

It is possible to model abstract and concrete interfaces using class models in UML. However class models don't provide an intuitive representation of the interface. UMLi provides an alternative
diagram notation for describing abstract interaction objects.

\image{6cm}{content/state_of_the_art/specification_languages/umli_login.png}{Login window modelled in UMLi}

Figure \ref{content/state_of_the_art/specification_languages/umli_login.png} shows an abstract user interface for a login windows modelled in UMLi. The upper container has four entities, \textit{username} and \textit{password} represent input controls while \textit{UsernameParam} and \textit{PasswordParam} are bindings to where the content of inputs will be stored. On the lower container are represented the actions of the window.

UMLi's user interface diagram consists of six constructors:
\begin{itemize}
\item \textbf{FreeContainers} rendered as dashed cubes. A \textit{FreeContainer} is a top-level interaction class that no other interaction class can contain.

\item \textbf{Containers} rendered as dashed cylinders. A Container is a mechanism that groups interaction classes other than \textit{FreeContainers}.

\item \textbf{Inputters} rendered as downward triangles. An \textit{Inputter} receives information from users.

\item \textbf{Editors} rendered as rhombi. An Editor facilitates the two-way exchange of information.

\item \textbf{Displayers} rendered as upward triangles. A Displayer sends information to users.

\item \textbf{ActionInvokers} rendered as right-pointing arrows. An \textit{ActionInvoker} receives direct instructions from users.
\end{itemize}

Tasks are usually represented in a tree notation in which leaf nodes are primitive tasks and non-leaf nodes group and describe relationships between their children nodes. There is a set of three essential features present in most task modelling languages:
\begin{itemize}
\item \textbf{Hierarchical decomposition}, high-level tasks systematically decompose into less abstract tasks.

\item \textbf{Temporal relationships}, the order in which a composite task's children are carried out depends on the parent's temporal relation.

\item \textbf{Primitive tasks}, the lowest-level nodes described in the task model are primitive tasks. An action task, for example, corresponds to an activity the application carries out. An interaction task involves some degree of human-computer interaction.
\end{itemize}

Use cases and activities in UMLi represent the notion of task with a set of features that include all the elementary ones mentioned above.

Using use cases and their scenarios, it's possible to elicit user interface functionalities required to let users achieve their goals. Possible ways to perform actions that support the functionalities elicited using use cases can be identified using activities. Therefore, mapping use cases into top level activities can help describe a set of interface functionalities similar to that described by
task models in other specification languages.

Use-case diagrams in UMLi are UML use-case diagrams. Activity diagrams in UMLi, however, extend activity diagrams in UML. UMLi provides a notation for a set of macros for activity diagrams that can be used to model behaviour categories usually observed in user interfaces: optional, order independent, and repeatable behaviours.

Using these macro notations, activity diagrams in UMLi can cope better with the tendency that activity diagrams have to become complex even when modelling the behaviour of simple user interfaces.

In order to represent relationships between models in UMLi object flows are used in activity diagrams to describe how to use class instances to perform actions in action states. By using object flows, it's possible to incorporate the notion of state into activity diagrams that are primarily used for modelling behaviour. In UMLi, it's also possible to use object flows to describe how to use interaction class instances. However, object flow states—rendered as dashed arrows connecting objects to action states—have specific semantics when associating interaction objects to activities and action states. UMLi specifies categories of object flow states specific to interaction objects:
\begin{itemize}
\item The \textbf{interacts} object flows relate primitive interaction objects to action states, which are primitive activities. They indicate that associated action states are responsible for interactions in which users invoke object operations or visualize the results of object operations.

\item The \textbf{presents} object flows relate \textit{FreeContainers} to activities and specify
that the associated \textit{FreeContainers} should be visible while the activities are active.

\item The \textbf{confirms} object flows relate \textit{ActionInvokers} to selection states and specify that selection states have finished normally.

\item The \textbf{cancels} object flows relate \textit{ActionInvokers} to composite activities or selection states and specify that activities or selection states have not finished normally and that the application flow of control should be rerouted to a previous state.

\item The \textbf{activates} object flows relate \textit{ActionInvokers} to other activities, thereby triggering the associated activities that start when an event occurs.
\end{itemize}

In \cite{User_Interface_Modeling_in_UMLi} is mentioned a case study specified both in UML and UMLi. A set of metrics were applied to each specification. The results of these metrics show that constructing and maintaining interactive system models should be simpler and easier in UMLi than in UML.
\chapter{UsiXML}

In section \ref{subsection:usixml} UsiXML was referred as part of the state of the art. In this chapter this subject will be studied in more depth. The focus of this chapter will not be just UsiXML as a specification language but the whole project behind it.

UsiXML's last version was released in February 14th, 2007. In April of 2009 was founded a consortium with the objective of lowering  the total application costs and development time by adding versatile context driven capabilities to UsiXML that would bring it far beyond the state of the art, up to the achievement of its standardisation.

The UsiXML consortium started with three main goals that would guide the entire project. The first goal is \textit{the UsiXML ``$\mu7$'' concept elicitation and promotion}. The $\mu7$ concept means that it should be possible to specify  a user interface for an interactive application that should support multi-device, multi-platform, multi-user, multi-linguality/culturality, multi-organisation, multicontext, or multi-modality capabilities.

The second goal is the \textit{development of the UsiXML language and the model-driven method}. The UsiXML language should guarantee interoperability, reusability, and maintainability of interactive applications developed. For this reason UsiXML is an open XML-compliant standard User Interface Description Language (UIDL). There should be models to cover every $\mu7$ aspects. Finally, UsiXML should define a flexible methodological framework that accommodates various development paths as found in organisations and that can be tailored to their specific needs.

The third goal is to \textit{set up development tools and demonstration of the validity on applications}. There should be a suite of software tools that support the methodological framework defined in goal 2 that can be later integrated or connected to available software environments. A UsiXML service will be defined and developed once so that this service can be deployed many times, especially for all environments requiring them to reduce the total cost of development. UsiXML will provide developers with various knowledge bases containing usability and accessibility guidelines that can be semi-automatically verified on any UsiXML-produced UI so as to guarantee a certain level of quality.

Section \ref{section:usixml_semantics} provides an in depth study of the UsiXML semantics. In section \ref{section:usixml_method} refers to the UsiXML engineering method for developing user interfaces.

\section{UsiXML semantics}
\label{section:usixml_semantics}

The core component of a user interface specified in UsiXML
consists of a uiModel, which is itself decomposed into several models. Not all models should be included. Only those models which are required for the particular user interface are included in a UsiXML file. Figure \ref{content/usixml/uml_spec.png} shows the components of UsiXML in an UML class diagram.

\image{\textwidth}{content/usixml/uml_spec.png}{UsiXML specification as an UML class diagram.}

The uiModel is the topmost superclass containing common features shared by all component models of a user interface. A uiModel may consist of a list of component models in any order and any number:
\begin{itemize}
\item \textbf{Transformation model}: contains a set of rules in order to enable a transformation of one specification to another.
\item \textbf{Domain model}: describes the classes of the objects manipulated by the users while interacting with the system.
\item \textbf{Task model}: describes the interactive task as viewed by the user interacting with the system. The task model is expressed according to the CTT specification \cite{ConcurTaskTrees_A_Diagrammatic_Notation_for_Specifying_Task_Models}.
\item \textbf{Abstract user interface model}: represents the view and behavior of the domain concepts and functions in platform independent way.
\item \textbf{Concrete user interface model}: represents a concretization of the abstract user interface model.
\item \textbf{Mapping model}: contains a series of related mappings between models or elements of models.
\item \textbf{Context model}: describes the three aspects of a context of use, which is a user carrying out an interactive task using a specific computing platform in a given surrounding environment.
\item \textbf{Resource model}: contains definitions of resources attached to abstract or concrete interaction objects.
\end{itemize}
The uiModel is also composed by a creation date, a schema version, a set of authors, a set of versions and a set of comments. A complete reference to UsiXML semantics can be found in \cite{UsiXML_USer_Interface_eXtensible_Markup_Language}.
\section{UsiXML method}
\label{section:usixml_method}

The MDE approach allows developing the UsiXML UI by transforming progressively the UsiXML models to obtain specifications that are detailed and precise enough to be rendered or transformed into code.

\input{./content/usixml/method/spem2}
\input{./content/usixml/method/spem4usixml}
\input{./content/usixml/method/forward_engineering_method}

\section{Conclusion}

The first tools in this chapter were Janus\cite{janus} and Its\cite{ITS}. These tools represent the beginning of model driven development of user interfaces and at the same time two different approaches to this subject. Developers using Janus will decrease the development time of their applications. This seems to be the main goal of the project. Because it only needs one input, the domain model, it doesn't take a lot of effort to create a user interface. Of course this also means that the generated user interface won't have the same quality as one built by hand (depending on the abilities of the developer) and this is the most criticized aspect of Janus. On the other hand, Its can produce high quality user interfaces. But the amount of work necessary to build something is huge compared to Janus. Its also provides a way to divide work through the development team which is a very interesting feature because it facilitates the implementation of different development methodologies alongside this tool.

By analysing these two projects one can infer that model driven development can be used with two objectives in mind:
\begin{enumerate}
\item to build low-cost prototypes of an application rapidly;
\item to enhance the quality of the final product.
\end{enumerate}

Janus is clearly a tool that can be used to rapidly create a prototype of an in development application that can be used to communicate with customers or other stake holders. Its is aimed at building better applications with large teams.

In section \ref{section:optimization_based_generation_of_interfaces} were introduced a couple of projects that use numeric optimization to generate user interfaces. The first project was Gadget\cite{gadget}. This is a framework that abstracts the programmer from the optimization itself. The problem with this project is that it's not a complete solution. Developers still have to produce evaluations which can be a tricky subject because it's not trivial to determine what properties make a user interface  ``good''. The second project, Supple\cite{supple}, it's more mature and complete. The objective behind Supple is to generate user interfaces at run time that are optimal to people with special needs.

Although Gadget is meant to simplify numeric optimization of user interfaces in a way that it could be used by any developer it still looks very complex to use this system.

Supple is a more complete system. It receives three types of input models:
\begin{itemize}
\item one related to the domain;
\item other related to the user;
\item and other related to the platform it's supposed to run on.
\end{itemize}

This makes Supple easier for developers but doesn't give them much control of the final user interface. The great advantage of this system is his capability to dynamically adapt to the needs of every user.

Section \ref{section:user_interface_patterns} covered the use of user interface patterns in the process of modelling a user interface. In this section IdealXML\cite{IdealXml_An_Interaction_Design_Tool, idealxml2} was looked in detail. IdeaXML is a tool where designers can, using several graphical notations, specify domain models, task models, abstract presentation models and mapping models between them. IdealXML manipulates a pattern repository, where patterns are organized following a hierarchical structure. This enables the usage of patterns at design time.

Unfortunately, like in most projects in this field, IdealXML is an academic project and it's not ready to be adopted by the industry.

Section \ref{section:specification_languages} covered ``Specification Languages''. The reason why this topic was approached in this document is that probably the most important problem with model driven development of user interfaces is the lack of standard specification language like UML.

The first language in this section is UMLi\cite{User_Interface_Modeling_in_UMLi}. This language is an extension to UML that supports every major step of model driven development of user interfaces. It only lacks support for the \textit{concrete user interface model}.

Lastly, UsiXML\cite{UsiXML_a_Language_Supporting_Multi-Path_Development_of_User_Interfaces}. UsiXML is a User Interface Description Language based on the Cameleon reference framework\cite{Calvary}. UsiXML supports every step in the Cameleon framework, including the concrete one, unlike UMLi.

UsiXML also ships with an engineering method and meta-method. Which allows development teams to easily adopt this technology and adapt the methodology according to their needs.

Not just because of all the advantages it provides but also because of it's vast community, UsiXML seems to have the potential to become the standard specification language for user interface specification.