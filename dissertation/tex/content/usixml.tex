\chapter{UsiXML}

In section \ref{subsection:usixml} UsiXML was referred as part of the state of the art. In this chapter this subject will be studied in more depth. The focus of this chapter will not be just UsiXML as a specification language but the whole project behind it.

UsiXML's last version was released in February 14th, 2007. In April of 2009 was founded a consortium with the objective of lowering  the total application costs and development time by adding versatile context driven capabilities to UsiXML that would bring it far beyond the state of the art, up to the achievement of its standardisation.

The UsiXML consortium started with three main goals that would guide the entire project. The first goal is \textit{the UsiXML ``$\mu7$'' concept elicitation and promotion}. The $\mu7$ concept means that it should be possible to specify  a user interface for an interactive application that should support multi-device, multi-platform, multi-user, multi-linguality/culturality, multi-organisation, multicontext, or multi-modality capabilities.

The second goal is the \textit{development of the UsiXML language and the model-driven method}. The UsiXML language should guarantee interoperability, reusability, and maintainability of interactive applications developed. For this reason UsiXML is an open XML-compliant standard User Interface Description Language (UIDL). There should be models to cover every $\mu7$ aspects. Finally, UsiXML should define a flexible methodological framework that accommodates various development paths as found in organisations and that can be tailored to their specific needs.

The third goal is to \textit{set up development tools and demonstration of the validity on applications}. There should be a suite of software tools that support the methodological framework defined in goal 2 that can be later integrated or connected to available software environments. A UsiXML service will be defined and developed once so that this service can be deployed many times, especially for all environments requiring them to reduce the total cost of development. UsiXML will provide developers with various knowledge bases containing usability and accessibility guidelines that can be semi-automatically verified on any UsiXML-produced UI so as to guarantee a certain level of quality.

Section \ref{section:usixml_semantics} provides an in depth study of the UsiXML semantics. In section \ref{section:usixml_method} refers to the UsiXML engineering method for developing user interfaces.

\section{UsiXML semantics}
\label{section:usixml_semantics}

The core component of a user interface specified in UsiXML
consists of a uiModel, which is itself decomposed into several models. Not all models should be included. Only those models which are required for the particular user interface are included in a UsiXML file. Figure \ref{content/usixml/uml_spec.png} shows the components of UsiXML in an UML class diagram.

\image{\textwidth}{content/usixml/uml_spec.png}{UsiXML specification as an UML class diagram.}

The uiModel is the topmost superclass containing common features shared by all component models of a user interface. A uiModel may consist of a list of component models in any order and any number, such as task model, a domain model, an abstract UI model, a concrete UI model, a mapping model, and context model, resource model, transformation model. The uiModel is also composed by a creation date, a schema version, a set of authors, a set of versions and a set of comments.
\section{UsiXML method}
\label{section:usixml_method}

The MDE approach allows developing the UsiXML UI by transforming progressively the UsiXML models to obtain specifications that are detailed and precise enough to be rendered or transformed into code.

\subsection{Spem 2.0}
\label{subsection:spem2}

SPEM is an OMG standard dedicated to software method modelling. The goal of SPEM is to propose minimal elements necessary to define any software and systems development method, without adding specific features to address particular domains. As a result, this meta-model supports a large range of development methods of different styles, levels of formalism, and life-cycle models.

SPEM is a UML profile, meaning that SPEM reuses UML whenever possible. Consequently, SPEM uses UML for various software model concepts presentations.

The current version of SPEM is 2.0 and it was completely reformulated from the previous one in order to separate the operational aspect of a method from the temporal aspect of a method. As shown in figure \ref{content/usixml/method/spem_base.png}, the SPEM 2.0 meta-model uses seven main meta-model packages:
\begin{itemize}
\item \textbf{Method Content} package describes the static aspect of a method; 
\item \textbf{Process Structure} and \textbf{Process Behaviour} packages describe the dynamic aspect of a method, \textbf{Process With Methods} package describes the link between these two aspects; 
\item \textbf{Core package} provides the common classes that are used in the different packages; \item \textbf{Method Plug-in} package describes the configuration of a method and Managed Content package describes the documentation of a method.
\end{itemize}

\image{12cm}{content/usixml/method/spem_base.png}{Structure of the SPEM 2.0 meta-model.}

In the following subsections we'll go into some detail for each package in order to get a better understanding of the SPEM 2.0 method meta-model.

\subsubsection{Core package}

The Core meta-model package contains abstract generalization classes that are specialized in the other meta-model packages. These abstract generalization classes are used to define
common properties of their specialized classes. The main elements of the core package are:
\begin{itemize}
\item The \textbf{Kind class}, that expresses a refined vocabulary specific to a method;
\item \textbf{Work Definition} is an abstract generalization class that represents the work being performed by a specific role, or the work performed throughout a life-cycle.
\item \textbf{Work Definition Parameter} is an abstract generalization class that represents parameters for Work Definitions.
\item \textbf{Work Definition Performer} is an abstract generalization class that represents the relationship of a work performer (role) to a Work Definition.
\end{itemize}

\subsubsection{Method content}

The method content meta-model package defines the core elements of every method (producer, work unit and work product) independently of any process or development project. It describes the specific development steps that are achieved by which roles with which resources and results, without specifying the placement of these steps within a specific
development life-cycle.

\image{16cm}{content/usixml/method/spem_mc.png}{Structure of the SPEM 2.0 method content meta-model package.}

Figure \ref{content/usixml/method/spem_mc.png} shows the Method Content meta-model package. The main classes of the Method content meta-model are:
\begin{itemize}
\item The \textbf{Task Definition} defines the work being performed by \textit{Roles Definition instances}. A Task is associated with input and output \textit{Work Products}.

\item A \textbf{Step} describes a meaningful and consistent part of the overall work described for
a \textit{Task Definition}. The collection of Steps defined for a \textit{Task Definition} represents all the work that should be done to achieve the overall development goal of the \textit{Task Definition}.

\item The \textbf{Work Product Definition} describes the product which is used, modified, and
produced by \textit{Task Definitions}.

\item The \textbf{Role Definition} designs a general reusable definition of an organizational role. It
defines a set of related skills, competencies, and responsibilities of an individual or a set of individuals. \textit{Roles} are used by \textit{Task Definitions} to define who performs them as well as to define a set of Work \textit{Product Definitions} they are responsible for.
\end{itemize}

\subsubsection{Process Structure}

The process structure meta-mode defines the structure of the method process. This package represents a process decomposed as a set of Activity classes that are linked to Role classes and Work Product classes. This structure is useful to express the fact that a life-cycle is composed by set of phases, and each phase is composed by set of activities.

\image{16cm}{content/usixml/method/spem_ps.png}{Main classes and associations of SPEM 2.0 process structure package package.}

Figure \ref{content/usixml/method/spem_ps.png} illustrates the Process Structure meta-model package. The most important classes of this meta-model are:

\begin{itemize}

\item A \textit{WorkDefinition} (coming from the Core package) is performed by a \textbf{Work
Definition Performer}, which is a role, and, through this role, by a \textit{Process Performer}.

\item A \textbf{Breakdown Element} is an abstract generalization class that defines a set of
properties available to all of its specializations.

\item A \textbf{Work Breakdown Element} provides specific properties for \textit{Breakdown Elements}
that represent work.

\item An \textbf{Activity} defines basic units of work within a process as well as a process itself. Every activity can represent a process in SPEM 2.0. It relates to \textit{Work Product Use} instances via instances of the \textit{Process Parameter} class and \textit{Role Use} instances via \textit{Process Performer} instances. An activity can be used by another activity, so that the structure of the source activity is copied into the target activity.

\item The \textbf{Role Use} represents a performer of an \textit{Activity} or a participant of the \textit{Activity}. A \textit{Role Use} is only specific to the context of an \textit{Activity}. It is not a general reusable definition of an organizational role like the \textit{Role Definition} of the Method Content package. 

\item A \textbf{Work Product Use} represents an input and/or output type for an \textit{Activity} or
represents a general participant of the \textit{Activity}.

\item The \textbf{Work Sequence} represents a relationship between two \textit{Work Breakdown Elements} in which one \textit{Work Breakdown Element} depends on the start or finish of another \textit{Work Breakdown Elements} in order to begin or end.

\end{itemize}
\subsection{Spem4UsiXML}
\label{subsection:spem4usixml}

\subsection{Forward engineering method}
\label{subsection:forward_engineering_method}
